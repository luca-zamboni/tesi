
\chapter{Conclusioni}

	La tesi descrive dei metodi per migliorare vari aspetti riguardo il riconoscimento di cartelli ed espone un metodo per calcolarne la distanza da esso. Siamo partiti da dei requisiti che deve avere l' algoritmo di riconoscimento descritti nel primo capitolo e abbiamo analizzato l'algoritmo esistente e fatto dei test di affidabilità e velocità. Dopo aver analizzato i test si è deciso come sviluppare il progetto per arrivare ai requisiti fondamentali per una applicazione real time. \'E stato deciso di concentrasi sul migliorare l'affidabilità e prestazioni. Inoltre è stato deciso di implementare una nuova funzione che permette di calcolare la distanza dal cartello. Infine nel quarto capitolo capitolo sono stati presentati i dati raccolti dagli esperimenti eseguiti per testate i suddetti requisiti finali che l'algoritmo deve avere.

	I risultato dei test è soddisfacente sotto il punto di vista della distanza di riconoscimento. Infatti, i nuovi cartelli con il rispettivo nuovo algoritmo di riconoscimento vengono letti con discreto successo fino a 6 metri di distanza rispetto ai 2 metri del codice di partenza. Inoltre, si è migliorato sotto l'aspetto di affidabilità cioè che un cartello non viene mai scambiato con un altro. Si sono ottenuti buoni risultati anche da parte della velocità di esecuzione. Se si cerca un cartello nell'intorno del punto sull'orizzonte si hanno dei miglioramenti in prestazioni di \textasciitilde $38$\% anche se il tempo per il calcolo è oneroso dato dal fatto che la libreria Line Segment Detection ha un elevato carico computazionale. Per concludere i test per il calcolo della distanza hanno dato ottimi risultati se il cartello è posto a meno di 3 metri dalla robot invece oltre i 3 metri la precisione cala drasticamente quindi è stato deciso per il momento di tenere in considerazione i dati relativi a cartelli posti a meno di 3 metri.

\section{Lavori futuri}

	Un aspetto importante che dovrebbe essere migliorato sono le prestazioni dell'algoritmo per il calcolo del Vanishing Point, in particolare la funzione per il calcolo dei segmenti nell'immagine la quale prende più tempo nell' algoritmo. Inoltre sarebbe opportuno testare la precisione di calcolo del Vanishing Point quando il robot sarà pronto per andare in strada. Un ulteriore step sarebbe quello di migliorare la precisione dell'algoritmo nel calcolo della distanza dal cartello posto a distanze superiori i 3 metri. In futuro, con lo sviluppo di schede embedded sempre più potenti e il conseguente aumento di prestazioni, congiunto all'utilizzo di GPGPU (Genral purpose GPU) che servono a parallelizzare i processi, consentirebbe di rendere più robusto l'algoritmo senza influire sui tempi di calcolo. Una scheda più potente potrebbe, per esempio, consentire di aumentare la risoluzione della telecamera mantenendo i tempi di calcolo contenuti.